% !TeX root =./x2.tex
% !TeX program = pdfpLaTeX

\newcommand{\ZU}{\marginpar{図を挿入}}

\chapter{準備}
\section{Intoroduction}
Quiver の表現の直既約分解に関する諸々について解説することが目標である.

\subsection{Quiver (矢筒, 箙(えびら))とは?}
(ラベルのついた)点(頂点)と,
それらを結ぶ(ラベルのついた)矢印(辺)
を集めたもの(図形)をQuiverと呼ぶ.
\begin{example}
  \ZU
  \begin{align*}
    \bullet_{1} \xrightarrow{\alpha} \bullet_{2}
  \end{align*}
\end{example}
\begin{example}
  \ZU
  \begin{align*}
    \bullet_{1} 
  \end{align*}
\end{example}
\begin{example}
  \ZU
  \begin{align*}
    \bullet_{1}
    \genfrac{}{}{0pt}{}{\xrightarrow{\alpha}}{\xrightarrow{\beta}}
    \bullet_{2}
  \end{align*}
\end{example}
\begin{example}
  \ZU
  \begin{align*}
    \genfrac{}{}{0pt}{}{\bullet_{1}}{\circlearrowleft_{\alpha}}
  \end{align*}
\end{example}

集合と写像の言葉で定義するなら,
\begin{enumerate}
\item $Q_0$: 集合 (頂点の集合)
\item $Q_1$: 集合 (辺(矢印)の集合)
\item $s\colon Q_1\to Q_0$: 写像. ($s(\alpha)$は辺$\alpha$の始点\footnote{start, source})
\item $t\colon Q_1\to Q_0$: 写像. ($t(\alpha)$は辺$\alpha$の終点\footnote{teminal, target})
\end{enumerate}
の4つのデータの組$(Q_0,Q_1,s,t)$がQuiver.
(多重辺, セルフループを許す有向グラフという言い方もする)


\begin{example}
  \ZU
  $Q_0=\Set{1,2,3}$,
  $Q_1=\Set{\alpha,\beta,\gamma,\delta}$,
  $s(\alpha)=1$,
  $t(\alpha)=2$,
  $s(\beta)=1$,
  $t(\beta)=2$,
  $s(\gamma)=2$,
  $t(\gamma)=1$,
  $s(\delta)=1$,
  $t(\delta)=1$
  は次の図で表されるものとなる.
\end{example}



\subsection{quiverの表現とは?}
$Q=(Q_0,Q_1,s,t)$をquiverとする.
\begin{enumerate}
\item
  $x\in Q_0$に対し,
  ($K$-)線型空間(有限次元ベクトル空間, 有限生成加群)$V_x$,
\item
  $\alpha\in Q_1$に対し,
  ($K$-)線型写像$f_\alpha\colon V_{s(\alpha)} \to V_{t(\alpha)}$
  が対応しているとき,
  その対応$(V_\bullet,f_\bullet)$を$Q$び
  ($K$-線型空間のカテゴリーへの)表現と呼ぶ.
\end{enumerate}

\begin{example}
  \ZU
  $Q_0=\Set{1}$,
  $Q_1=\Set{\alpha}$,
  $s(\alpha)=1$,
  $t(\alpha)=1$
  とする.
  このとき,
  例えば,
  \begin{align*}
    V_1=\CC^2&\\
    f_\alpha\colon V_1&\to V_1\\
    \begin{pmatrix}a\\b\end{pmatrix}
      &\mapsto
    \begin{pmatrix}b\\a\end{pmatrix}
  \end{align*}
  とすれば,
  $(V_\bullet,f_\bullet)$は$Q$の表現である.
  例えば,
  \begin{align*}
    V_1=\CC^2&\\
    f'_\alpha\colon V_1&\to V_1\\
    \begin{pmatrix}a\\b\end{pmatrix}
      &\mapsto
    \begin{pmatrix}a\\0\end{pmatrix}
  \end{align*}
  とすれば,
  $(V_\bullet,f'_\bullet)$は$Q$の表現である.
\end{example}

\begin{example}
  \ZU
  $Q_0=\Set{1,2}$,
  $Q_1=\Set{\alpha}$,
  $s(\alpha)=1$,
  $t(\alpha)=2$
  とする.
  このとき,
  例えば,
  \begin{align*}
    V_1=\CC^2&\\
    V_2=\CC^2&\\
    f_\alpha\colon V_1&\to V_2\\
    \begin{pmatrix}a\\b\end{pmatrix}
      &\mapsto
    \begin{pmatrix}b\\a\end{pmatrix}
  \end{align*}
  とすれば,
  $(V_\bullet,f_\bullet)$は$Q$の表現である.
  例えば,
  \begin{align*}
    V_1=\CC^2&\\
    V_2=\CC^2&\\
    f'_\alpha\colon V_1&\to V_2\\
    \begin{pmatrix}a\\b\end{pmatrix}
      &\mapsto
    \begin{pmatrix}a\\0\end{pmatrix}
  \end{align*}
  とすれば,
  $(V_\bullet,f'_\bullet)$は$Q$の表現である.
\end{example}

\begin{example}
  \ZU
  $Q_0=\Set{1,2}$,
  $Q_1=\Set{\alpha,\beta}$,
  $s(\alpha)=s(\beta)=1$,
  $t(\alpha)=t(\alpha)=2$
  とする.
  このとき,
  例えば,
  \begin{align*}
    V_1=\CC^2&\\
    V_2=\CC^2&\\
    f_\alpha\colon V_1&\to V_2\\
    \begin{pmatrix}a\\b\end{pmatrix}
      &\mapsto
    \begin{pmatrix}a\\0\end{pmatrix}\\
    f_\beta\colon V_1&\to V_2\\
    \begin{pmatrix}a\\b\end{pmatrix}
      &\mapsto
    \begin{pmatrix}0\\b\end{pmatrix}
  \end{align*}
  とすれば,
  $(V_\bullet,f_\bullet)$は$Q$の表現である.
\end{example}

\begin{example}
  $Q=(Q_0,Q_1,s,t)$をquiverとする.
  このとき,
  \begin{enumerate}
  \item 各$x\in Q_0$に対して, $V_x=\Set{0}$ ($0$)
  \item 各$\alpha\in Q_1$に対して,
    \begin{align*}
      f_\alpha = \underline{0} \colon \Set{0} &\to \Set{0}\\
      0&\mapsto 0
    \end{align*}
  \end{enumerate}
  とすると
  $(V_\bullet,f_\bullet)$は$Q$の表現である.
  これを零表現と呼ぶ.
\end{example}

\begin{example}
  $Q=(Q_0,Q_1,s,t)$をquiverとする.
  $(V_\bullet, f_\bullet)$,
  $(U_\bullet, g_\bullet)$
  を$Q$の表現とする.
  \begin{enumerate}
  \item
    $x\in Q_0$に対し,
    $W_\bullet=V_x\oplus U_x$
  \item
    $\alpha\in Q_1$に対し, 
    \begin{align*}
      h_\alpha = f_\alpha\oplus g_\alpha \colon
      V_{s(\alpha)}\oplus U_{s(\alpha)}&\to
      V_{t(\alpha)}\oplus U_{t(\alpha)}
      \\
      (v,u)&\mapsto (f_\alpha(v),g_\alpha(u))
    \end{align*}
  \end{enumerate}
    とおくと,
    $(W_\bullet,h_\bullet)$は$Q$の表現である.
    これを,
    $(V_\bullet, f_\bullet)$と
    $(U_\bullet, g_\bullet)$の
    直和とよび,
    $(V_\bullet, f_\bullet)\oplus (U_\bullet, g_\bullet)$
    で表す.
  \end{example}

\subsection{Quiverの表現の構造について}

$Q$をquiverとする.
$Q$の表現$(V_\bullet,f_\bullet)$を理解するにはどうしたらよいか?

一つの答え:
より``小さい''表現に分解する.
\begin{align*}
  (V_\bullet,f_\bullet)=
  (V'_\bullet,f'_\bullet)\oplus (V''_\bullet,f''_\bullet)
\end{align*}
みたいに直和に分解していく.
$\dim(V'_x),\dim(V''_x)\leq\dim(V_x)$
なのだから, $V'_x$, $V''_x$の方が簡単に違いない.
分解を頑張ってこれ以上分解できないところまで分解して,
\begin{align*}
(V_\bullet,f_\bullet)=
  (V^{(1)}_\bullet,f^{(1)}_\bullet)\oplus (V^{(2)}_\bullet,f^{(2)}_\bullet)
  \oplus \cdots
\end{align*}
と書けたら,
`これ以上分解できない表現達' (直既約表現)
がわかっていれば,
$(V_\bullet,f_\bullet)$
はわかったことになるはず.
(直既約表現の方が$(V_\bullet,f_\bullet)$よりも簡単なはず.)

問題:
\begin{enumerate}
\item
  \label{basicquestion:ks}
  表現は直既約表現の直和として書けるのか?
\item
  \label{basicquestion:ar}
  与えられた直既約表現の直和に分解する方法(アルゴリズム)?
\item
  \label{basicquestion:gt}
  $Q$の直既約表現は何種類くらいあるのか?
  有限個/無限個?
\end{enumerate}

\Cref{basicquestion:ks}について:
Krull-Schimidtの定理 ($\CC$上ならOK)

\Cref{basicquestion:ar}について:
(非アルゴリズム的\footnote{無限回のステップからなる手続きを含む}な)
一般論としてはAuslander--Reiten理論が知られている.
アルゴリズム的な方法は一部のQuiverに対して与えられている.
(行列の行基本変形の延長)

\Cref{basicquestion:gt}について:
Gabrielの定理.
次は同値:
\begin{enumerate}
\item
  $Q$の直既約表現が有限種類しかない.
\item
  $Q$は(simply laced) Dynkin図形に向きをつけたものののみ.
\end{enumerate}
ただし, (simply laced) Dynkin図形とは,
\ZU
$A_n$, $D_n$, $E_n$
のこと.


実は, \ZU
\begin{itemize}
\item
  $Q=\genfrac{}{}{0pt}{}{\bullet}{\circlearrowleft}$のときには,
  $Q$の表現の直既約分解はジョルダン標準形を求めることに相当する.
\item
  $Q=\bullet\to\bullet$のときには,
  $Q$の表現の直既約分解はランク標準形(スミス標準形)($\begin{pmatrix}E&O\\O&O\end{pmatrix}$)を求めることに相当する.
\end{itemize}
これらについては前半で復習する.


\subsection{$A_n$型quiverの表現と関連する話}
\subsubsection{$A_n$型quiverの表現について}
\begin{align*}
Q=\bullet_1\xrightarrow{\alpha_1}\bullet_2\xrightarrow{\alpha_2}\bullet_3\cdots \bullet_{n-1}\xrightarrow{\alpha_{n-1}}\bullet_n
\end{align*}
について考える.
$1\leq i\leq j\leq n$に対し,
区間表現$I[i,j]=(V_\bullet,f_\bullet)$を次で定める:
\begin{enumerate}
\item
  $x\in Q_0$に対し,
  \begin{align*}
    V_x=
    \begin{cases}
      \CC & (i\leq x \leq j)\\
      \Set{0} & (\text{otherwise}).
    \end{cases}
  \end{align*}
\item
  $\alpha_k\in Q_1$に対し,
  \begin{align*}
    f_{\alpha_k}=
    \begin{cases}
      \underline{0}\colon\Set{0} \to \Set{0} & (1\leq k < i-1)\\
      \underline{0}\colon\Set{0} \to \CC & (k=i-1)\\
      \id_\CC\colon\CC \to \CC & (i\leq k < j)\\
      \underline{0}\colon\CC \to \Set{0} & (k=j)\\
      \underline{0}\colon\Set{0} \to \Set{0} & (j+1\leq k).
    \end{cases}
  \end{align*}  
\end{enumerate}
つまり,
\begin{align*}
  I[i,j]\colon
  0\xrightarrow{\underline{0}}
  0\xrightarrow{\underline{0}} \cdots \xrightarrow{\underline{0}}
  0\xrightarrow{\underline{0}}
  \underbrace{\CC}_{i} \xrightarrow{\id_\CC}
  \CC \xrightarrow{\id_\CC} \cdots
  \underbrace{\CC}_{j} \xrightarrow{\underline{0}}
  0\xrightarrow{\underline{0}}
  0\xrightarrow{\underline{0}} \cdots \xrightarrow{\underline{0}}
  0.
\end{align*}
実は,
$Q$の既約表現は, $I[i,j]$ ($1\leq i\leq j \leq n$)で
(同型を除き)全部であることが知られている.
つまり, $Q$の表現$(V_\bullet,f_\bullet)$は,
\begin{align*}
  (V_\bullet,f_\bullet)
  &=I[i_1,j_1]\oplus\cdots\oplus I[i_N,j_N]
\end{align*}
という形で書ける.

\subsubsection{Persistent Homology}
ホモロジー:
幾何的対象 $\leadsto$ ホモロジー群
\ZU
`穴の個数'を調べる道具.



時刻$t$に合わせて変化する図形$X_t$を考える.
\begin{example}
  \label{ex:fundex:ph}
\ZU
中心は動かず時刻に合わせて
半径が大きくなる円たち.
\end{example}

各時刻でのホモロジー$H(X_t)$が得られる.
つまり時刻$t$に合わせて変化するホモロジー群(ベクトル空間)
$\leadsto$ パーシステントホモロジー.

\Cref{ex:fundex:ph}の様に
\begin{align*}
  t\leq t' \implies X_t \subset X_{t'}
\end{align*}
という条件をみたいしていたら,
\begin{align*}
  X_t&\injectto X_{t'}\\
  x&\mapsto x
\end{align*}
から, 線型写像
\begin{align*}
  H(X_{t'})&\to H(X_t)\\
\end{align*}
を作ることができる,
よって,
$t_1<\cdots <t_n$を取ると,
\begin{align*}
  H(X_1)\to H(X_2)\to\cdots
\end{align*}
という$A_n$型の表現が得られる.
この表現の直既約分解
\begin{align*}
  I[i_1,j_1]\oplus \cdots\oplus I[i_N,j_N]
\end{align*}
を考える.
$I[i_k,j_k]$に含まれるホモロジー類は,
時刻$i_k$に現れ時刻$j_k$に消えることをがわる.
標語的な言い方をすると,
`時刻$i_k$で生まれ
時刻$j_k$で消える
``穴''がある'
ことを意味している.
したがって,
`穴' の個数だけでなく,
その生存時刻(大きさ)がわかる.

$\leadsto$
\begin{itemize}
\item
  この情報を使ってデータ解析をするというのが流行っている.
\item
  例えば, $t$だけではなく,
  2つのパラメータ$(t,s)$などで変化する図形のときには,
  別のQuiverの表現.
\end{itemize}


\subsubsection{Leschets性}
多項式環$A=K[x_1,\ldots,x_n]$を考える.
$f\in A$ 
に現れる単項式が$k$次のもののみであるとき,
$f$は$k$次斉次多項式であるという.
$A_k$で$k$次斉次多項式を集めた集合とする.
(ただし, $0\in A_k$と約束する.)

$f_1,\ldots, f_n\in A$
とし, いずれも斉次多項式であるとする.
$I$を$f_1,\ldots, f_n$で生成されるイデアルとする.
つまり
$I=\Braket{f_1,\ldots,f_m}_{A}$
とする.
剰余環
$R=A/I$
について考える.
$R_k=\Set{\overline{f}\in R|f\in A_k}$
とおくと,
$R_i$は$K$-線型空間であり,
$R=\bigoplus_{k=0}^{infty}A_k$
と書ける.
一般には$R$は無限次元の$K$-線型空間であるが,
ここでは,
アルチン環であることを仮定し,
$R=\bigoplus_{k=0}^{s}A_k$, $A_s\neq \Set{0}$
と書けるとする.

$\overline{l}\in R_1$とする.
このとき,
\begin{align*}
  \times \overline{l}\colon
R_{k}&\to R_{k+1}  \\
\overline{f}&\mapsto \overline{fl}
\end{align*}
は$K$-線型写像であり,
\begin{align*}
  R_0\xrightarrow{\times\overline{l}}
  R_1\xrightarrow{\times\overline{l}}
  R_1\xrightarrow{\times\overline{l}}\cdots
\end{align*}
という表現が得られる.
この表現の直既約分解
\begin{align*}
  I[i_1,j_1]\oplus \cdots\oplus I[i_N,j_N]
\end{align*}
として,
\begin{align*}
  i_1\leq i_2 \leq \cdots\leq i_N \leq j_N \leq j_{N-1} \leq \cdots \leq j_1
\end{align*}
となっているとき,
$\overline{l}$は$R$の広義Lefschetz元と呼ぶ.

Lefschetz元を持つ環は,
コホモロジー環
などと言った面白い環であることが多い.

与えられた環がLefschetz元を持つか?
もしもつなら, Lefschetz元は何か?
という問題を考えることができる.

\section{今後の予定}
\begin{enumerate}
\item 線型代数の復習---行列の標準形 (証明はせず事実のみを紹介する)
  \begin{enumerate}
  \item 行基本変形
  \item 対角化
  \item Jordan標準形
  \end{enumerate}
\item Quiverの表現に関する概説---直既約表現
  \begin{enumerate}
  \item 定義
  \item $\bullet \to \bullet$の表現
  \item $\genfrac{}{}{0pt}{}{\bullet}{\circlearrowleft}$の表現
  \item finite type/infinite fype
  \end{enumerate}
\item
  関連する話題.
  \begin{enumerate}
  \item パーシステントホモロジー
  \item Lefschetz性
  \end{enumerate}
\end{enumerate}

目標は
\begin{enumerate}
\item 行列の対角化とかができる.
\item Quiverという言葉を知っている.
\end{enumerate}
ARはやらない.

\tableofcontents
\chapter{行列についての復習}
\section{行列の用語と記号}
